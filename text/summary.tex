%	12-sep-89 : summary created in LaTeX		- PJT
%	16-nov-89 : updated for new HOSTTYPE environments - PJT
%	12-jun-94 : updated for NEMOHOST and new 2.2 mosaic based export - PJT
%%%%%%%%%%%%%%%%%%%%%%%%%%%%%%%%%%%%%%%%%%%%%%%%%%%%%%%%%%%%%%%%%%%%%%%%%%%%%%
%%                                                                          %%
%% To print this LaTex document:                                            %%
%%                                                                          %%
%%	latex summary							    %%
%%	dvipr summary				                            %%
%%                                                                          %%
%%                              Peter Teuben                                %%
%%                                                                          %%
%%%%%%%%%%%%%%%%%%%%%%%%%%%%%%%%%%%%%%%%%%%%%%%%%%%%%%%%%%%%%%%%%%%%%%%%%%%%%%

\documentstyle[]{report}

\title{NEMO USERS/PROGRAMMERS MANUAL - SUMMARY SHEET}

\date{Alpha Version 2.2-MD \\
 June 12 1994 \\
 Document revised: \today\ by P. Teuben}

%%%%%%%%%%%%%%%%%%%%%%%%%%%%%%%%%%%%%%%%%%%%%%%%%%%%%%%%%%%%%%%%%%%%%%%%%%%
\begin{document}
\setlength{\parindent}{0pt}
\setlength{\parskip}{2.5mm}

\centerline{SUMMARY - USER INSTALL}

To {\underline {\bf use}} NEMO make sure the following is present in
your default startup file(s) {\tt .cshrc}/{\tt .login}:

\begin{itemize}
\item Setting the root directory
    of NEMO: 
\begin{verbatim}
	setenv NEMO /home/nemo
\end{verbatim}
    Of course the rootname of NEMO on your system will probably differ 
    from the one listed in this example.

\item  Executing the startup script for NEMO, which sets other 
    environment variables etc: 
\begin{verbatim}
	source $NEMO/NEMORC
\end{verbatim}

\item Add NEMOs bin directory (NEMOBIN, which is set by NEMORC)
    to your search path. 
    An example could be:
\begin{verbatim}
 	set path=( . $NEMOBIN $path )
\end{verbatim}
\end{itemize}

A second method to load NEMO is by demand only, (so-called dynamic loading)
by adding the following alias anywhere in your .cshrc file:
\begin{verbatim}
 	alias nemo 'source /home/nemo/nemo_start'
\end{verbatim}

\newpage
\centerline{SUMMARY - USER INTERFACES}

Most NEMO programs are invoked from the commandline, just like any other
UNIX program, with a series of parameters of the form `{\it
keyword=value}'.  We call these {\bf program keywords}, as the names of
the keywords are unique to a program.  The {\it keyword=} part is not
needed in case the original order of the keywords is used.  Once a
keyword is specifically named, all remaining keywords need the keyword
name too, e.g.  `{\tt prognam val1 val2 key6=val6 key7=val7}'.  The
value of a keyword can also be obtained from an 'include' file by using
the {\it keyword=@file} construct. 

There also exist a fixed set of {\bf system keywords}, which are defined
for \underline{every} NEMO program. They are:

\begin{itemize}
\item {\bf help} Sets help level to program\footnote{inline version: help='?'}
and can aid in command
completion. Consist of a {\it numeric} and {\it alpha} part. The 
{\it numeric} part is a mask of individual help levels which are powers
of 2:
  \begin{itemize}
  \item[1.] restore and save keywords in keyword file {\tt prognam.def}.
  \item[2.] interactive prompting per keyword
  \item[4.] menu interface using environment variable {\bf EDITOR}.
  \item[8.] curses based menu interface.
  \item[16.] reserved
  \end{itemize}
For example, {\tt help=5} would no only use the menu interface, but
also restore and save the keyword file.
   
The {\it alpha} part typically shows 
inline\footnote{as opposed to online help, which are the manual pages}
help:
  \begin{itemize}
  \item[?]  list of options of help keyword (this list)
  \item[h]  list keywords plus help string (requires new defv-format)
  \item[a]  show the {\it keyword=value} strings.
  \item[p,k] show the {\it keyword} strings.
  \item[d,v] show the {\it value} strings.
  \item[n]  add newline between {\it keyword=value} strings
  \item[t]  output in {\tt exectool} format
  \item[q]  quit
  \end{itemize}
The {\bf HELP}
environment variable will be used if keyword is not used.

% \item {\bf host} Define the remote host on which this program is
% to be executed. The {\tt rsh} command must have been installed,
% and the remote host must have the same disk name. No environment
% variable used.

\item {\bf debug} Define debug output level between 0 and 9. Negative
numbers are allowed, and should produce absolutely no warning messages.
The higher the level, the more warnings and debug output. The
routine {\it dprintf(3NEMO)} uses the debug value. The {\bf DEBUG}
environment variable will be used if keyword is not used.

\item{\bf error} Bypass a number of fatal {\it error(3NEMO)} calls. It's really
cheating, use at your own risk. The {\bf ERROR}
environment variable will be used if keyword is not used.

\item {\bf yapp} Define graphics device, if applicable, for the program.
The  {\bf YAPP} environment variable will be used if keyword is not used.

\end{itemize}

Reminder: environment variables can be set/reset using the shell commands
{\tt setenv} and {\tt unsetenv}.

\newpage
\centerline{SUMMARY - ADVANCED USER INTERFACES}

There are two more advanced (sometimes considered more user friendly)
pieces of user interfaces to NEMO that are worth mentioning:

\begin{itemize}

\item {\bf miriad:} Within NEMO the {\tt miriad} shell can also be
used to present a user interface that 
bears some similarity to AIPS. Usage:

\begin{verbatim}
	% miriad -b $NEMOBIN -p $NEMODOC
\end{verbatim}


\item {\bf nemoman:} The command {\tt nemoman} is a front-end
to a variety of manual browsers. Currently implemented are 
the {\tt xman} (default, or -x flag) 
and {\tt tkman} (using -tk flag). Your local system manager must
have installed the programs themselves, {\tt nemoman} merely calls
them with the correct parameters. Examples:

\begin{verbatim}
	% mirman
	% mirman -tk
	% manlaser snapplot
\end{verbatim}

The last command pertains to be smart about the NEMO manual pages, and
searches for and prints the manual page for snapplot on your default
laserprinter. Works fine on SUN workstations, but may need patchwork.


\end{itemize}

\end{document}

