%	10-dec-90 : created in LaTeX		- PJT
%	12-jun-92 : updates
%%%%%%%%%%%%%%%%%%%%%%%%%%%%%%%%%%%%%%%%%%%%%%%%%%%%%%%%%%%%%%%%%%%%%%%%%%%%%%
%%                                                                          %%
%% To print this LaTex document:                                            %%
%%                                                                          %%
%%	latex faq							    %%
%%	dvipr faq (or whatever)			                            %%
%%                                                                          %%
%%                              Peter Teuben                                %%
%%                                                                          %%
%%%%%%%%%%%%%%%%%%%%%%%%%%%%%%%%%%%%%%%%%%%%%%%%%%%%%%%%%%%%%%%%%%%%%%%%%%%%%%

\documentstyle[]{report}

\title{NEMO USERS/PROGRAMMERS MANUAL - SUMMARY SHEET}

\date{Version 2.1-UMD \\
 Summer 1992 \\
 Document revised: \today\ by P. Teuben}

%%%%%%%%%%%%%%%%%%%%%%%%%%%%%%%%%%%%%%%%%%%%%%%%%%%%%%%%%%%%%%%%%%%%%%%%%%%
\begin{document}
\setlength{\parindent}{0pt}
\setlength{\parskip}{2.5mm}

\centerline{FREQUENTLY ASKED QUESTIONS (FAQ)}

\begin{itemize}
\item What is NEMO?
{\bf Answer:} NEMO is a dynamics toolbox, 
for users as well as programmers. It is a set of programs and libraries,
which serve to exend your UNIX environment.

\item Does NEMO run on my machine?
{\bf Answer:} Designed to run under UNIX, some modules have been ported to
MS-DOS and VMS. We do not recommend nor formally support this.
\item In what language is NEMO written?
{\bf Answer:} Most of the source code
(currently about 10,000 lines) is written in K\&R
C. There is some support to interface Fortran modules.

\item Who do I contact for more information?
{\bf Answer:} teuben@astro.umd.edu, who should also receive
problems, bugs and enhancements..



\item Do I need to know Unix in order to use NEMO? 
{\bf Answer:} In short: Yes.  You should know how to log in, things like
how to use a windowing system and how it applies to your current
terminal (if you log into a computer from a remote terminal, you cannot
always use it's graphics capabilities), you ought to know what files and
directories are, how they are created (either by editor or by programs),
how you cleanup your directories etc. It will probably help a lot to
use UNIX pipes, utilities like grep and awk, head and tail etc. 

\item If NEMO is installed on my system, how can I use it?
{\bf Answer:} Read Appendix A, although you do need to know
in what directory NEMO has been installed, i.e. the value
to set the NEMO environment variable to.

\item What kind of programs can I use in NEMO
{\bf Answer:} The manual pages {\it index.1} and {\it programs.8}.
Use the script {\tt manlaser}, or something equivalent,
to get hardcopies.
The first manual page is an alphabetical index of NEMO programs
and utilities, the second on a listing by topic.

\item Can files produced by NEMO on one type of
computer be used on another?
{\bf Answer:} Yes and No. Some are interchangable (sun, dec)
as long as they use IEEE for floating point and twos complement
for integers. Byte ordering is not important, although you will
see messages that data is being read in byte swapped order. There
is potential trouble when a program reads two different endian
programs.

\item 
{\bf Answer:}

\item 
{\bf Answer:}

\item 
{\bf Answer:}

\item 
{\bf Answer:}

\item 
{\bf Answer:}

\item 
{\bf Answer:}

\item 
{\bf Answer:}

\item 
{\bf Answer:}


\end{itemize}

\end{document}



