\magnification=1200
\font\big=cmr10 scaled \magstep2
\vglue 3.0cm
\hsize=6truein
\vsize=8.5truein
\centerline {\big NBODY2}
\bigskip
\bigskip
\centerline {AHMAD--COHEN N-BODY CODE}
\bigskip
\centerline {Sverre J. Aarseth, Institute of Astronomy, Cambridge}
\bigskip
\bigskip
\centerline {INTRODUCTION}
\bigskip

  This program is based on the direct integration method of A. Ahmad and L.
 Cohen
 in {\it J. Comp. Phys.} 12, 389, 1973.  The procedure has been modified
 slightly
 and the improvements are described in the 1985 book article on direct
 methods for N-body simulations ({\it Multiple Time Scales}, eds.
 Brackbill and Cohen).  This article also makes comparisons with the
 standard polynomial method (program NBODY1).  The present version
 does not contain regularization of close two-body encounters
 and is therefore best suited for soft potentials, dissipative
 effects (i.e. tidal fields or supernova mass loss) or a small number
 of crossing times.  A regularized version of the AC method is also
 available and is described in separate write-ups (programs NBODY4 and
 NBODY5).

  The program is written in FORTRAN IV and has been developed on the
 IBM 370/165.  The integrator itself contains about 470 FORTRAN statements and
 the whole package is about 1395 lines of EBCDIC code (the VAX
 version is in ASCII code).  This includes the generation of initial
 conditions, an output routine and general comment cards.  Logical units
 5 and 6 are used for input and output, respectively, whereas units
 1, 2 and 3 denote optional tape/disc facilities.

  The alphanumerical list of all COMMON variables is given below.  Then follow
 tables of input parameters, options and diagnostic output counters.  A brief
 description of each routine is included.  Together with the algorithm write-up
 and the internal comments this should provide an understanding of how the
 program works and enable
 some simple modifications to be made.  The last section contains a discussion
 of the relevant input parameters.

  Some saving of core storage may be achieved by overlaying different parts of
 the program.  The routines can be arranged into segments as follows: (1) MAIN,
 MYDUMP; (2) START, DATA; (3) MAINPT; (4) INTGRT.  Segment 1 is the root phase;
 the other segments are formed by the linkage editor using appropriate
 include statements naming the routines.

  The present version contains sufficient allocation of COMMON variables for
 integrating up to 2500 particles.  Modification to other systems is readily
 achieved by a global edit, setting new values of NMAX and LMAX in
 the PARAMETER statement which precedes COMMON.  Thus any value
 of NMAX and LMAX can be specified without further changes.  The maximum
 size of the neighbour list is discussed at the end.  The array
 NLIST(500) should be large enough for N $<$ 10000 since its
 membership is stabilized on NNBMAX (note
 explicit tests on the membership at end of routine START and
 just below label 4 of INTGRT).
\bigskip
\bigskip
\centerline {COMMON VARIABLES}
\medskip
\hrule
\medskip
\settabs\+\indent&NSTEPNX\quad&\cr
           \+&A&     General purpose working space. \cr
           \+&BE&    Total energy.  BE(1)=initial, BE(2)=previous,
 BE(3)=current. \cr
           \+&BODY&    Mass of an individual particle. \cr
           \+&CMR&     Centre of mass coordinates and scalar value. \cr
           \+&CMRDOT&  Centre of mass velocity components and scalar value. \cr
           \+&DELTAT&  Constant time interval for output. \cr
           \+&DTLIST&  Interval for updating the time-step list. \cr
           \+&D1&      First divided difference of the irregular force. \cr
           \+&D1R&     First divided difference of the regular force. \cr
           \+&D2&      Second divided difference of the irregular force. \cr
           \+&D2R&     Second divided difference of the regular force. \cr
           \+&D3&      Third divided difference of the irregular force. \cr
           \+&D3R&     Third divided difference of the regular force. \cr
           \+&EPS2&    Soft potential parameter (square is stored). \cr
           \+&ETAI&    Time-step parameter for irregular force polynomial. \cr
           \+&ETAR&    Time-step parameter for the regular force polynomial. \cr
           \+&F&       One-half the total force at the last irregular step. \cr
           \+&FDOT&    One-sixth the total force derivative at last irregular
 step. \cr
           \+&FI&      Irregular force due to the neighbours. \cr
           \+&FR&      Regular force due to the distant particles. \cr
           \+&F1&      Irregular or regular force on current particle. \cr
           \+&F1DOT&   First force derivative of current particle (both types).
 \cr
           \+&F2DOT&   Second force derivative of current particle (both types).
 \cr
           \+&F3DOT&   Third force derivative of current particle (both types).
 \cr
           \+&F4DOT&   Fourth force derivative of current particle (both types).
 \cr
           \+&ILIST&   Current list of neighbours. \cr
           \+&JLIST&   Current list of receding neighbours in shell within
 1.26$\ast$RS. \cr
           \+&KZ&      Options for alternative paths at run time (see table).
 \cr
           \+&LIST&    Sequential neighbour list (LIST(1,I) = membership). \cr
           \+&N&       Total particle number. \cr
           \+&NDUMP&   Control variable for automatic energy check procedure.
 \cr
           \+&NFIX&    Output frequency of binaries or data save (options 3 and
 6). \cr
           \+&NLIST&   Time-step list for determining next body to be
 integrated. \cr
           \+&NNBMAX&  Maximum number of neighbours (allow one extra with option
 5). \cr
           \+&NPRINT&  Output counter (reset to zero when NPRINT = NFIX). \cr
           \+&NSTEPN&  Counters for diagnostic output information (see table).
 \cr
           \+&NSTEPS&  Time-step counter for calling computer clock (reset to
 zero). \cr
           \+&ONE2&    The constant 0.5. \cr
           \+&ONE3&    The constant 1.0/3.0. \cr
           \+&ONE4&    The constant 0.25. \cr
           \+&ONE5&    The constant 0.2. \cr
           \+&ONE6&    The constant 1.0/6.0. \cr
           \+&ONE12&   The constant 1.0/12.0. \cr
           \+&QE&      Relative energy tolerance for error control (option 2).
 \cr
           \+&RS&      Radius of the standard neighbour sphere. \cr
           \+&RSCALE&  Half-mass radius determined from the potential energy.
 \cr
           \+&SMIN&    Close encounter time-step for retention of old
 neighbours. \cr
           \+&STEP&    Time-step for irregular force polynomial. \cr
           \+&STEPR&   Time-step for regular force polynomial. \cr
           \+&TCRIT&   Termination time in scaled units. \cr
           \+&TIME&    Physical integration time in scaled units. \cr
           \+&TLIMIT&  Maximum computing time in minutes (used with the timer).
 \cr
           \+&TLIST&   Time for next updating of the time-step list. \cr
           \+&TNEXT&   Time for next output. \cr
           \+&T0&      Time of last irregular force calculation. \cr
           \+&T0R&     Time of last regular force calculation. \cr
           \+&T1&      Time of first backwards irregular force calculation. \cr
           \+&T1R&     Time of first backwards regular force calculation. \cr
           \+&T2&      Time of second backwards irregular force calculation. \cr
           \+&T2R&     Time of second backwards regular force calculation. \cr
           \+&T3&      Time of third backwards irregular force calculation. \cr
           \+&T3R&     Time of third backwards regular force calculation. \cr
           \+&X&       Coordinates at current time. \cr
           \+&XDOT&    Velocities at beginning of step and current values at
 output. \cr
           \+&X0&      Coordinates at the beginning of a step. \cr
           \+&X0DOT&   Velocities at the beginning of a step. \cr
\medskip
\hrule
\bigskip
\bigskip
\vfill\eject
\centerline {INPUT PARAMETERS}
\medskip
\hrule
\medskip
           \+&KSTART*&  Values 1,2,3 indicate new case, restart, or modified
 restart. \cr
           \+&TCOMP&   Maximum computing time in minutes (stored in TLIMIT). \cr
           \+&N*&       Total particle number. \cr
           \+&NFIX&    Output frequency of binaries or data save (options 3 and
 6). \cr
           \+&NNBMAX&  Maximum number of neighbours (option 5 needs one more).
 \cr
           \+&ETAI*&    Time-step parameter for irregular force polynomial. \cr
           \+&ETAR&    Time-step parameter for regular force polynomial. \cr
           \+&RS0&     Initial radius of neighbour sphere. \cr
           \+&DELTAT&  Output time interval in units of the crossing time. \cr
           \+&TCRIT&   Termination time in units of the crossing time. \cr
           \+&QE&      Energy tolerance (restart if DE/KE $>$ $5\ast$QE and
 KZ(2) $>$ 0). \cr
           \+&SMIN&    Close encounter time-step for neighbour retention (option
 5). \cr
           \+&CUTOFF&  Soft potential parameter (stored as square in EPS2). \cr
           \+&KZ(J)*&   Non-zero options for alternative paths (see separate
 table). \cr
\medskip
\hrule
\medskip
\+& & *)New input line (free format). \cr
\bigskip
\bigskip
\settabs\+\indent&XXX\quad&\cr
\centerline {OPTIONS KZ(J)}
\medskip
\hrule
\medskip
           \+&^^1&  COMMON save on unit 1 if TCOMP $>$ TLIMIT or if TIME $>$
 TCRIT. \cr
           \+&^^2&  COMMON save on unit 2 for restart if DE/E $>$ 5$\ast$QE. \cr
           \+&^^3&  Binary search every NFIX output time. \cr
           \+&^^4&  Skips full predictor loop if neighbour number $>$ KZ(4) and
 KZ(4) $>$ 0. \cr
           \+&^^5&  Retains small step neighbours inside 2$\ast$RS (STEP(J) $<$
 SMIN). \cr
           \+&^^6&  Essential data written to unit 3 at output time (frequency
 NFIX). \cr
           \+&^^7&  All bodies listed at output time. \cr
           \+&^^9&  First 5 bodies listed at output time. \cr
          \+&17&  Modification of accuracy parameters ETAI and ETAR using
 tolerance. \cr
\medskip
\hrule
\bigskip
\bigskip
\centerline {OUTPUT COUNTERS NSTEPN(J)}
\medskip
\hrule
\medskip
           \+&^^1&  Irregular integration steps. \cr
           \+&^^2&  Regular integration steps. \cr
           \+&^^3&  Full predictor loops. \cr
           \+&^^4&  Force polynomial corrections. \cr
           \+&^^5&  Too many neighbours with standard criterion. \cr
           \+&^^6&  No neighbours inside 1.26 times the basic radius. \cr
          \+&13&  Retained neighbours using time-step criterion SMIN. \cr
          \+&14&  Large F3DOT corrections not included if also STEP $<$ SMIN.
 \cr
          \+&18&  No neighbours with standard criterion. \cr
\medskip
\hrule
\vfill\eject
\centerline {MAIN}
\bigskip

  This is the main routine which controls the calling sequence.  First it reads
  two decision variables, i.e. KSTART and TCOMP.  KSTART = 1,2,3 denotes new
 case,
 restart from COMMON dump on tape/disc or restart with modified parameters,
 respectively.  The local variable TCOMP specifies the maximum computing time
 in minutes to be used for the run.  It is subsequently stored in the COMMON
 variable TLIMIT since otherwise an over-write would occur at
 restart time.  Timing
 procedures are usually installation dependent and the VAX system statements
 CALL LIB\$INIT used for initialization and CALL LIB\$STAT giving the elapsed
 time may therefore have to be replaced.  The latter is used near the end
 of routines MAINPT and INTGRT (displayed between asterisks).

 Note that all the COMMON variables are read back at restart time and the start
 routine is therefore only needed at TIME = 0.  Once control passes to routine
 INTGRT a return does not occur until the next output time denoted by variable
 TNEXT.  The program terminates either in routine MAINPT or in INTGRT, depending
 on the conditions TIME $>$ TCRIT or TCOMP $>$ TLIMIT, respectively.  In either
 case
 all COMMON variables are saved if option 1 is activated and the appropriate
 tape/disc unit has been allocated.
\bigskip
\bigskip
\centerline {MYDUMP}
\bigskip
 A tape/disc routine is provided for writing or reading the COMMON
 variables.  The general
             call statement is CALL MYDUMP(I,J).  If I = 0, all COMMON blocks
 are read from logical unit J, whereas I = non-zero is used for saving
 COMMON.  The restart
 procedure requires that all important variables are stored in COMMON.  Several
 labelled COMMON blocks are used for organizational convenience.

 The size of COMMON blocks is consistent with the other routines.  At
 present the storage requirement is 260$\ast$NMAX + 2$\ast$LMAX$\ast$NMAX + 1548
 bytes,
 where LMAX denotes the maximum size of the neighbour list array for each
 particle (take NNBMAX = LMAX -- 2).  Note that the size of BLOCK1 is
 40$\ast$NMAX + 4
 bytes for the single precision version but twice as much is needed when
 using the present double precision statements for
 X, X0, X0DOT, T0 and TIME.
\bigskip
\bigskip
\centerline {START}
\bigskip
 This routine reads the basic parameters and performs the initialization for
 the integrator.  Relevant counters and variables are first set to zero and the
 fractional constants ONE2, etc are specified.  The reason for introducing these
 constants is to save time by avoiding divisions (only ONE3, ONE6 and
 ONE12 are implemented).

 Four data cards are required for a run.  All input is read in free
 format as READ (5,$\ast$) for convenience but this can readily be
 changed.  The relevant variables are listed
 in the input table and the choice of integration parameters is
 discussed at the end of this write-up.  Initial values for the variables
 BODY(I), X(K,I), XDOT(K,I) where I = 1,2,...,N and K = 1,2,3 must be specified
 next.  These initial conditions are supplied by routine DATA which will
 normally be written by the user.  After the call
     to routine DATA the force polynomials are obtained by the method of
 explicit differencing which only requires the masses, coordinates and
 velocities (see the 1985 algorithm for details).

 There are two main loops, both
 requiring N$\ast\ast$2 operations.  The force and first derivative due to
 neighbours
 and distant particles are evaluated separately in one loop.  The choice of
 neighbour sphere radius RS for each particle depends on the density as well as
 the maximum permitted neighbour number (NNBMAX).  If the system is not
 homogeneous it is desirable to modify the individual neighbour
 radius (local variable RS2) at the beginning of the loop, otherwise it will be
 set to the input value RS0.  There is an automatic
 reduction/increase of the neighbour sphere if the membership is not in the
 range (1, NNBMAX -- 1).

 A second main loop obtains the second and third force derivatives for the
 irregular and regular force components.  This procedure requires the total
 force and first derivative (F and FDOT) which are also used by the integrator
 for
 coordinate predictions.  To save time the second and third derivatives for
 the regular component are only evaluated inside a distance 3$\ast$RS since more
 distant particles do not contribute significantly, i.e. F2DOT scales as
 1/R$\ast\ast$5.

  Individual integration steps for the irregular and regular force polynomials
 are determined from a relative criterion of the type
 STEP = (ETA$\ast$(F$\ast$F2DOT + FDOT$\ast\ast$2)/(FDOT$\ast$F3DOT +
 F2DOT$\ast\ast$2))$\ast\ast$0.5
 where the parameter ETA controls the rate of convergence.  Note that this
 expression is also well behaved for particles at rest.  The time-step loop
 initializes the times used for the backwards difference scheme (four variables
 for each type).  This loop also
 sets X0(K,I) = X(K,I) and X0DOT(K,I) = XDOT(K,I).  These variables denote the
 values at the beginning of a step and are only updated at the end of each
 individual time-step, whereas the variables X(K,I) which represent current
 coordinates are over-written by the predictor.  Also note that the variables
 XDOT(K,I) are used to represent current velocities in the output
 routine.  Finally one-half the total force and one-sixth the total force
 derivative is set in F(K,I) and FDOT(K,I), respectively.  This permits a fast
 coordinate prediction to be made.

 Members of the time-step list are determined in the final loop.  The starting
 value of DTLIST is taken as one-half the average time-step, subject to a
 membership in the permitted range.  The array
 NLIST contains the index of all particles to be treated during the interval
 (0, DTLIST), with NLIST(1) holding the membership number.  If the initial
 irregular time-steps are equal, NLIST should be of size N + 1 (note
 the explicit test near the end and also in routine INTGRT).
\bigskip
\bigskip
\centerline {DATA}
\bigskip

 Provides the initial conditions for the starting routine and will normally
 be written by the user.  Random numbers within (0, +1.0) are obtained by the
  VAX library function RAN(KKK).  The
 variables which must be specified are BODY(I),
 X(K,I), XDOT(K,I) for each particle I = 1,2,...,N with K = 1,2,3 denoting
 components of coordinates and velocities.  Unless there are special reasons
 it is recommended to initialize the coordinates and velocities with respect
 to the centre of mass rest frame.  The standard scaling convention for cluster
 calculations is to take total mass = N and adjust the coordinates and
 velocities such that the total energy = --N$\ast\ast$2/4.  This gives a virial
 scale
 factor of unity at equilibrium and a crossing time = 2$\ast$(2/N)$\ast\ast$0.5.
  However,
 the program works for any set of consistent units.
\bigskip
\bigskip
\centerline {MAINPT}
\bigskip

  This routine deals with output and error control.  The current
 coordinates and velocities are first predicted to highest order.  The next loop
 which has N$\ast$(N--1)/2 terms evaluates the potential energy.  The density
 centre (based on maximum neighbour density) is set in CMR(1,2,3).
 Other quantities calculated are kinetic energy,
                centre of mass coordinates, angular momentum about the Z-axis
 and the average neighbour number (denoted by $<$NNB$>$).  If non-zero,
 option 4 is updated to KZ(4) = $<$NNB$>$.

   The main output lines are mostly self-explanatory.  Here CMR
 gives the current c.m. value of coordinates, DE is the relative
 energy error (evaluated with respect to the kinetic energy), E
 is total energy, TCR denotes the time in units of the crossing
 time, AZ is Z-component of angular momentum, $<$R$>$ is
 approximate half-mass radius, and RHOMAX is maximum density
 contrast. The six first counters NSTEPN(J) are also printed.
 One line is printed for each particle if
 option 7 is activated and this is reduced to only the first five particles if
      option 9 is non-zero.  Option 3 permits a binary search with a frequency
 specified by the input parameter NFIX.  Each line contains index of components,
 mass of components, relative binding energy per unit mass, semi-major axis,
 mean motion, separation, central distance, eccentricity, and neighbour
 number for the first component.

 A facility for automatic error control is included (option 2) and works as
 follows.  A restart from the last output time is made with reduced time-step
 parameters if the relative energy (absolute value) exceeds 5$\ast$QE, where QE
 is
 specified at input.  If option 17 is also activated and the relative error is
 in the range (QE, 5$\ast$QE) the time-step parameters are merely reduced by an
 appropriate amount, whereas ETAI and ETAR are increased slightly if the error
 is $<$ 0.2$\ast$QE.  Note that the program terminates after one unsuccessful
 restart
 using the COMMON variable NDUMP.  As a safety precaution the program also
 terminates if the error exceeds 5$\ast$QE and option 2 is zero.

     Options 1 and 2 may be used independently if desired.  However, the most
 flexible procedure is to have both options non-zero.  In this case the
 calculation can be halted at any time.  Alternatively option 2 may be used
 for both error control and termination at output time by moving the test
 TCOMP $>$ TLIMIT from INTGRT to MAINPT.  If there are no
 numerical problems (i.e. fairly soft potential) it is sufficient to use
 option 1 by itself.

 Option 17 may be used independently of option 2.  The adopted strategy can
 easily be modified if it is not satisfactory.  Most runs to date have been
 made without this facility, relying instead on reading new time-step
 parameters at restart time if appropriate.

 Routine INTGRT uses the density centre CMR(K) for including a radial
 velocity modification of the neighbour sphere.  If the system does not
 have a natural density centre it is better to set CMR(K) = 0.0 after
 obtaining the scalar value CMR(K), or suppress the RIDOT term in
 INTGRT (near label 68).

 The velocities XDOT which are over-written must be restored before the
 COMMON save in order that a correct integration restart can be made.  Also note
 that the predictor uses XDOT rather than X0DOT to save time in case the double
 precision version is used.  The data on both unit 1 and 2 is
 temporary, being over-written each time.  It is most convenient to accumulate
 the basic data such as TIME, BODY(I), X(K,I), XDOT(K,I) on a separate tape/disc
 unit.  The best place for doing this is just before setting
 XDOT(K,I) = X0DOT(K,I), i.e. when XDOT represent current velocities and the
 solution is acceptable.  An example of writing data to unit 3 is included
 with option 6 (frequency NFIX).  When using disc instead of tape the particle
 number should be
 written first as a separate record since otherwise the arrays cannot be
 read.  Note that X(K,I) is stored as a double precision number
 unless first copied into a local array.
\bigskip
\bigskip
\centerline {INTGRT}
\bigskip

The integration procedure is based on a predictor using three force differences
 and a fourth-order corrector.  The Ahmad-Cohen scheme represents the force
 field due to neighbours and distant particles by separate polynomials.  The
 main idea is to evaluate the neighbour force at frequent intervals and
 obtain the total force by adding the predicted contribution from distant
 particles.  The regular force component is updated less frequently and this
 results in a considerable saving.  Consistent corrections to the polynomial
 derivatives are made whenever there is a change of neighbours.  The neighbour
 strategy depends on available core storage and may be designed to suit
 individual problems.  The two-dimensional arrays are written explicitly inside
 loops for increased efficiency, i.e. X(1,I), X(2,I), X(3,I).  Note that
 on a scalar machine like VAX the array X(K,I)
 is faster than X(I,K) which was used previously.

  The first part determines the next body to be integrated (denoted by index I
 throughout).  Note that the current value of the time is set by
 TIME = T0(I) + STEP(I).  Only those particles which
 fall due within an interval DTLIST are examined each time and a new list is
 formed in the array NLIST at the end of this interval.  The variable DTLIST is
 stabilized on a membership in the range (0.5$\ast$NNBMAX, NNBMAX) which is
 close to the optimum.  The size of NLIST should be chosen somewhat
 larger than NNBMAX (which is usually about N$\ast\ast$0.5) to include bunching
 of time-steps.  Adjust the explicit test on the permitted membership
 accordingly.

            The integration cycle starts with coordinate predictions to
 order FDOT for either neighbours or all particles,
 depending on whether the regular force component
 can be extrapolated or needs to be evaluated again.  Old and new
 neighbour numbers are denoted by the local variables NNB0 and NNB,
 respectively.  The local
 variable IR is used as an indicator for the relevant case (i.e. IR = 0 or
  IR = 1) which is determined by comparing TIME + STEP(I) with
 T0R(I) + STEPR(I).  It is possible
 to save time by restricting the full predictor loop to particles with small
 neighbour numbers since other particles are predicted more frequently anyway,
 i.e. there is a multiplicity factor of NNB in the predictor.  Option
 4 is used for this purpose with the convention that if non-zero it specifies
 the limiting neighbour number for the full predictor loop (i.e. skip full
 prediction if NNB0 $>$ KZ(4)).  A neighbour
 prediction is performed instead if the full loop is skipped.  Note that this is
 the only instance where the actual value of an option is used
 (KZ(4) = $<$NNB$>$ is set in MAINPT).  This time
 saving is not recommended for non-equilibrium systems or if massive bodies are
 present.

 The total force polynomial for body I is now formed to highest order
 (i.e. F3DOT) and new coordinates and velocities are obtained.  This is followed
 by a force summation over the neighbours.  After updating the relevant times
 T0, T1, T2, T3, new divided differences are set for the
 irregular force component.  The fourth difference formed from the old and new
 third difference is used as a corrector (so-called semi-iteration) since
 experiments show that there is a useful gain in accuracy at no extra cost.  At
    this stage there are two paths depending on the control indicator.  If
 IR = 0, the current regular force and first derivative are extrapolated from
 the third-order polynomial and added to the irregular force and extrapolated
 first derivative to form the total force and first derivative which are
 used in the subsequent predictions.  Control then
              passes to the end of the cycle where a new irregular integration
 step is obtained, whereupon another cycle is started.  The alternative path
 described below requires a determination of the total force as well as a
 new neighbour search and associated correction procedure.

 The total force loop evaluates the irregular and regular components
 separately and stores the new neighbours in array ILIST.  Approaching
 particles between RS and 1.26$\ast$RS are included in the neighbour field if
 the
 impact parameter $<$ RS.  Other particles in the outer sphere are
 stored in array JLIST and will be selected if ILIST is empty, whereas a new
 search is only made if JLIST is also empty.  Note the general convention that
 the first list location is reserved for the membership number, although
 ILIST(1) and JLIST(1) are not used here.  Unless
 NNB $\leq$ NNBMAX the neighbour sphere radius is reduced and some members are
 removed while their force contributions are updated consistently.  The
 value of NNBMAX should allow for the retention of one old neighbour at
 a later stage.  The neighbour strategy is based on a simple distance criterion
 combined with an impact parameter test for approaching particles in the outer
 shell (of equal volume to the inner sphere).  It may
 be advantageous to introduce a mass-dependent criterion if the
 masses differ by many orders of magnitude.  This would require modification of
 all three distance tests in the regular force loop and also in the loop which
 reduces the neighbour number.

 One good strategy for choosing the neighbour sphere radius uses the
 square root of the number density contrast
 RHOS = 2$\ast$(NNB/N)$\ast$(RSCALE/RS(I))$\ast\ast$3.  The half-mass radius
 RSCALE is
 calculated from the potential energy
 in routine MAINPT.  It should be replaced if not appropriate (see $<$R$>$
 in the output).  The present modification
 of RS aims at five neighbours for a density contrast of unity and
 NNBMAX = 28.  The maximum permitted volume change of the neighbour
 sphere is limited to 25 \% either way.  Moreover,
 the neighbour sphere is increased by the smallest radial
 velocity factor RSDOT$\ast$STEPR(I) if NNB $<$ 4 and all neighbours are moving
 into the outer shell.

 An efficient search procedure is used to find the loss and gain of
  neighbours at the same time, which is faster than that employed
  previously.  If the step is suitably small
 (i.e. STEP(J) $<$ SMIN) a candidate for removal may be retained to avoid large
 derivative corrections.  This is implemented if the particle is inside twice
 the neighbour radius, in which case the relevant force components are
 updated.  This procedure may be skipped for fairly soft potentials
 (take option KZ(5) = 0).

 The next part deals with the regular force differences.  The procedure
 is almost identical to that for the irregular component.  The
 differences are first evaluated for a constant (i.e. old) neighbour
 field.  This is
         achieved by subtracting the change of neighbour force due to the old
 and new members.  The current total force
 and first derivative are then set and the fourth-order corrector
 is included.  Derivative corrections due to the loss and gain of
 neighbours are now
 accumulated.  These terms are calculated in a separate dual purpose
 part at the end of the routine.  Note that the loss of neighbours
 must be known before evaluating the regular differences because
 of the retention procedure.  If there has been a change of neighbours the
 membership list is updated and consistent corrections performed to the force
 polynomials.  Finally new regular and irregular time-steps are set using the
 composite expression given in the next section.  If desired, a new integration
 cycle may now be entered at label 1.

 Control passes temporarily to the main routine at every output time, followed
 by a call to MAINPT before returning.  The
 routine ends with a timing check after a specified number of integration
 steps (both types counted equally).  Provided that option 1 is non-zero, the
 COMMON blocks are saved on unit 1 if the calculation does not finish in the
 specified time (i.e. TCOMP $>$ TLIMIT).  The VAX timer LIB\$STAT is used for
 this
 purpose.  Intermediate COMMON saves on unit 1 may also be introduced to
 safeguard against system failure during long runs.
\bigskip
\bigskip
\centerline {CHOICE OF INPUT PARAMETERS}
\bigskip

 The choice of integration parameters depends on the requirements as well as
 the available precision.  In principle the accuracy is only limited by the
 computer word-length since the procedure does not require any
 approximations.  Note however that the most distant particles (i.e. those
 outside 3$\ast$RS) are neglected in the initialization of F2DOT and F3DOT for
 the sake
 of efficiency.  Moreover, very large corrections to F3DOT are omitted to
 prevent
 a severe time-step reduction for the regular component (only if STEP(J) $<$
 SMIN).  For 32-bit word machines (i.e. IBM 370 and VAX) it is
 advisable to define COMMON /BLOCK1/ in double precision together with a few
 local variables in routine INTGRT.  This
 leads to a significant improvement in accuracy at little extra cost since
 the force calculations are the most time-consuming part.  By changing the
 square root functions to DSQRT the program should also run in full double
 precision if desired but a few
 numerical constants may then have to be converted (compiler dependent).

 There are only two primary integration parameters, i.e. ETAI and ETAR.  A
 simple time-step expression of the type STEP = (ETA$\ast$F/F2DOT)$\ast\ast$0.5
 was used for
 many calculations (taking ETAI = 0.02, ETAR = 0.06).  It has now been replaced
 by a more efficient one which employs all the Taylor series derivatives, i.e.
 STEP = (ETA$\ast$(F$\ast$F2DOT + FDOT$\ast\ast$2)/(FDOT$\ast$F3DOT +
 F2DOT$\ast\ast$2))$\ast\ast$0.5.  This criterion
 is more sensitive and is particularly
 advantageous if the potential is very soft or if a repulsive force is
 included.  The recommended choice of time-step parameters for the composite
 expression is ETAI = 0.03, ETAR = 0.08.  For increased accuracy try
 ETAI = 0.02, ETAR = 0.06.  Note that the choice of ETAI and ETAR should be
 consistent with the machine accuracy.  The larger value permitted for ETAR
 is partly due to the procedure of reducing the regular time-step to coincide
 with the start of an irregular step.  There is no restriction on the softening
 parameter and mass-dependence can readily be included.

 The optimum value of the neighbour sphere is problem dependent but the
 machine time is not a sensitive function of the ratio NNB/N, i.e. there is a
 shallow minimum because of compensating factors.  An average neighbour
 number of about five is sufficient for N = 100, increasing to
 20 for N = 4000.  Choose a maximum neighbour number of at least
 NNBMAX = 10 + N$\ast\ast$0.5, subject to NNBMAX $<$ LMAX -- 1 (or $<$ LMAX
 without
 option 5).  In order to have about NNB neighbours for a constant
 density system, take an initial neighbour radius
 RS0 = (NNB/N)$\ast\ast$0.33$\ast$$<$R$>$ ($<$R$>$ = half-mass radius).
 Individual values
 may be introduced if
 there are significant density variations.  The present version with
 LIST(NMAX,LMAX) permits NNBMAX = LMAX -- 2 but NNBMAX = LMAX -- 1 is allowed
 if the SMIN retention procedure is
 not used.  Note that the standard neighbour search loop now accepts
 NNB = NNBMAX but one extra member may be added later if option 5 is non-zero.

 The energy tolerance QE may be used to modify the time-step parameters ETAI
  and ETAR during the calculation (options 2 and 17).  A typical value
 for an output interval of one crossing time is QE = 2.0E--04.  This gives
 rise to a slightly smaller rms relative energy error.  The time-step for
 retention of close encounter neighbours (with option 5) may be taken
 as SMIN =
 0.04$\ast$(4/N)$\ast\ast$1.5$\ast$($<$R$>$$\ast\ast$3/$<$M$>$)$\ast\ast$0.5.
 This corresponds to twice
 the standard value for starting a two-body regularization.  Set a small dummy
 value for SMIN with large softening (EPS $>$ 4$<$R$>$/N).  Option 4 is used
 to skip full predictor loops for neighbour numbers $>$ KZ(4).  If
 KZ(4) $>$ 0 it is set to the average value ($<$NNB$>$) every output.  About
 3/4 of the regular steps can be skipped, unless a massive body
 is present (check counter NSTEPN(3)).

 One possible modification is to include the tidal field in a star cluster
 calculation.  The case of a circular cluster orbit in the galactic plane is
 conveniently
              studied in rotating coordinates.  The relevant equations of motion
 are formulated in {\it Bulletin Astronomique} 2, 47, 1967.  Add the Coriolis
 force
 to the irregular component and the tidal force to the regular
 component.  Consistent derivatives must be included in routine START.  The
 latter are obtained using the explicit derivative principle.  Note that the
 Coriolis force should be added to the irregular force at two different places
 in routine INTGRT (i.e. near labels 25 and 60).  The current tidal
 force should be added to F1(K) just after label 60 of routine INTGRT.  It
 then becomes part of the regular force FR(K,I) and total force F(K,I).  The
 case of a linearized tidal field permits an energy integral to be used.
\bye

